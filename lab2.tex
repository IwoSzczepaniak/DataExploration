\documentclass{article}

% Language setting
% Replace `english' with e.g. `spanish' to change the document language
\usepackage[polish]{babel}

% Set page size and margins
% Replace `letterpaper' with `a4paper' for UK/EU standard size
\usepackage[a4paper,top=2cm,bottom=2cm,left=3cm,right=3cm,marginparwidth=1.75cm]{geometry}
\usepackage[T1]{fontenc}

% Useful packages
\usepackage{amsmath}
\usepackage{graphicx}
\usepackage[colorlinks=true, allcolors=blue]{hyperref}
\usepackage{float}

\title{Eksploracja Danych - zadanie 2}
\author{Iwo Szczepaniak}

\begin{document}
\maketitle

\section{Przetwarzanie zbioru xy-001}
Dane zostały załadowane i przekształcone przy użyciu VectorAssembler, który konwertuje kolumnę X do formatu wektora cech wymaganego przez algorytm regresji. Kolumna wynikowa została nazwana "features".

\subsection{Ładowanie i przetwarzanie wstępne}
\begin{verbatim}
+--------+----------+
|       X|         Y|
+--------+----------+
|0.411194|-59.938274|
|0.549662|-72.006761|
|0.860093|-68.979336|
| 1.27504|  32.07157|
|2.202931|-91.531894|
+--------+----------+
only showing top 5 rows

root
 |-- X: double (nullable = true)
 |-- Y: double (nullable = true)
\end{verbatim}

\subsection{Regresja}
Wykorzystano algorytm LinearRegression z następującymi parametrami:
\begin{itemize}
    \item MaxIter: 10
    \item RegParam: 0.3
    \item ElasticNetParam: 0.8
\end{itemize}

\begin{verbatim}
Coefficients: [-73.66234946939554]
Intercept: 635.3543312755129
numIterations: 2
objectiveHistory: [0.5000000000000002,0.40546708248417435,0.04089257570641288]
+-------------------+
|          residuals|
+-------------------+
| -665.0030891477942|
|  -666.871697941466|
| -640.9771961333321|
| -509.3603192080548|
| -564.6131520965479|
|-449.37832961747904|
+-------------------+

MSE: 88011.40099115501
RMSE: 296.6671552281361
MAE: 259.7641459358794
r2: 0.9331616863674084
\end{verbatim}

\begin{figure}
    \centering
    \includegraphics[width=0.75\linewidth]{image231.png}
    \caption{Enter Caption}
    \label{fig:enter-label}
\end{figure}

\subsection{Wykres funkcji i danych}
Utworzono funkcję do wizualizacji:
\begin{itemize}
    \item punktów danych treningowych
    \item funkcji regresji
    \item prawdziwej funkcji ($f_{true}$)
\end{itemize}

\begin{figure}[H]
    \centering
    \includegraphics[width=0.75\linewidth]{image_2.png}
    \caption{Funkcja rysująca wykresy}
    \label{fig:enter-label}
\end{figure}

\subsection{Wpływ parametrów regularyzacji}
Przeanalizowano wpływ parametru regularyzacji na wyniki modelu. Poniżej przedstawiono szczegółową analizę dla różnych wartości parametru RegParam.

\subsubsection{RegParam = 0.0 (brak regularyzacji)}
Model bez regularyzacji osiągnął następujące wyniki:
\begin{itemize}
    \item Współczynnik: -74.3271
    \item Wyraz wolny: 652.8037
    \item MSE: 87913.0977
    \item R²: 0.9332
\end{itemize}

\begin{figure}[H]
    \centering
    \begin{minipage}{0.45\textwidth}
        \includegraphics[width=\linewidth]{Screenshot 2025-03-24 at 21.46.38.png}
        \caption{Historia funkcji straty dla RegParam = 0.0}
    \end{minipage}
    \hfill
    \begin{minipage}{0.45\textwidth}
        \includegraphics[width=\linewidth]{Screenshot 2025-03-24 at 21.46.41.png}
        \caption{Regresja i punkty dla RegParam = 0.0}
    \end{minipage}
\end{figure}

\subsubsection{RegParam = 10.0}
Przy średniej regularyzacji otrzymano:
\begin{itemize}
    \item Współczynnik: -73.6623
    \item Wyraz wolny: 635.3543
    \item MSE: 88011.4010
    \item R²: 0.9332
\end{itemize}

\begin{figure}[H]
    \centering
    \begin{minipage}{0.45\textwidth}
        \includegraphics[width=\linewidth]{Screenshot 2025-03-24 at 21.47.14.png}
        \caption{Historia funkcji straty dla RegParam = 10.0}
    \end{minipage}
    \hfill
    \begin{minipage}{0.45\textwidth}
        \includegraphics[width=\linewidth]{Screenshot 2025-03-24 at 21.47.17.png}
        \caption{Regresja i punkty dla RegParam = 10.0}
    \end{minipage}
\end{figure}

\subsubsection{RegParam = 20.0}
Przy zwiększonej regularyzacji:
\begin{itemize}
    \item Współczynnik: -73.6623
    \item Wyraz wolny: 635.3543
    \item MSE: 88011.4010
    \item R²: 0.9332
\end{itemize}

\begin{figure}[H]
    \centering
    \begin{minipage}{0.45\textwidth}
        \includegraphics[width=\linewidth]{Screenshot 2025-03-24 at 21.47.37.png}
        \caption{Historia funkcji straty dla RegParam = 20.0}
    \end{minipage}
    \hfill
    \begin{minipage}{0.45\textwidth}
        \includegraphics[width=\linewidth]{Screenshot 2025-03-24 at 21.47.39.png}
        \caption{Regresja i punkty dla RegParam = 20.0}
    \end{minipage}
\end{figure}

\subsubsection{RegParam = 50.0}
Przy silnej regularyzacji:
\begin{itemize}
    \item Współczynnik: -72.9999
    \item Wyraz wolny: 617.9655
    \item MSE: 88304.9462
    \item R²: 0.9329
\end{itemize}

\begin{figure}[H]
    \centering
    \begin{minipage}{0.45\textwidth}
        \includegraphics[width=\linewidth]{Screenshot 2025-03-24 at 21.47.52.png}
        \caption{Historia funkcji straty dla RegParam = 50.0}
    \end{minipage}
    \hfill
    \begin{minipage}{0.45\textwidth}
        \includegraphics[width=\linewidth]{Screenshot 2025-03-24 at 21.47.55.png}
        \caption{Regresja i punkty dla RegParam = 50.0}
    \end{minipage}
\end{figure}

\subsubsection{RegParam = 100.0}
Przy bardzo silnej regularyzacji:
\begin{itemize}
    \item Współczynnik: -71.0262
    \item Wyraz wolny: 566.1597
    \item MSE: 90336.8273
    \item R²: 0.9314
\end{itemize}

\begin{figure}[H]
    \centering
    \begin{minipage}{0.45\textwidth}
        \includegraphics[width=\linewidth]{Screenshot 2025-03-24 at 21.48.21.png}
        \caption{Historia funkcji straty dla RegParam = 100.0}
    \end{minipage}
    \hfill
    \begin{minipage}{0.45\textwidth}
        \includegraphics[width=\linewidth]{Screenshot 2025-03-24 at 21.48.23.png}
        \caption{Regresja i punkty dla RegParam = 100.0}
    \end{minipage}
\end{figure}

\section{Analiza pozostałych zbiorów danych}

\subsection{Analiza poszczególnych zbiorów}

\subsubsection{Zbiór xy-001.csv}
Analiza pierwszego zbioru danych wykazała następujące metryki:
\begin{itemize}
    \item MSE: 98.5211
    \item RMSE: 9.9258
    \item MAE: 8.3756
    \item R²: 0.9289
\end{itemize}

\begin{figure}[H]
    \centering
    \begin{minipage}{0.45\textwidth}
        \includegraphics[width=\linewidth]{Screenshot 2025-03-24 at 22.14.02.png}
        \caption{Historia funkcji straty dla xy-001}
    \end{minipage}
    \hfill
    \begin{minipage}{0.45\textwidth}
        \includegraphics[width=\linewidth]{Screenshot 2025-03-24 at 22.14.07.png}
        \caption{Regresja i punkty dla xy-001}
    \end{minipage}
\end{figure}

\subsubsection{Zbiór xy-002.csv}
Analiza drugiego zbioru danych:
\begin{itemize}
    \item MSE: 88011.4010
    \item RMSE: 296.6672
    \item MAE: 259.7641
    \item R²: 0.9332
\end{itemize}

\begin{figure}[H]
    \centering
    \begin{minipage}{0.45\textwidth}
        \includegraphics[width=\linewidth]{Screenshot 2025-03-24 at 22.14.39.png}
        \caption{Historia funkcji straty dla xy-002}
    \end{minipage}
    \hfill
    \begin{minipage}{0.45\textwidth}
        \includegraphics[width=\linewidth]{Screenshot 2025-03-24 at 22.14.47.png}
        \caption{Regresja i punkty dla xy-002}
    \end{minipage}
\end{figure}

\subsubsection{Zbiór xy-003.csv}
Wyniki dla trzeciego zbioru danych:
\begin{itemize}
    \item MSE: 155761.2500
    \item RMSE: 394.6660
    \item MAE: 318.8897
    \item R²: 0.8352
\end{itemize}

\begin{figure}[H]
    \centering
    \begin{minipage}{0.45\textwidth}
        \includegraphics[width=\linewidth]{Screenshot 2025-03-24 at 22.15.03.png}
        \caption{Historia funkcji straty dla xy-003}
    \end{minipage}
    \hfill
    \begin{minipage}{0.45\textwidth}
        \includegraphics[width=\linewidth]{Screenshot 2025-03-24 at 22.15.04.png}
        \caption{Regresja i punkty dla xy-003}
    \end{minipage}
\end{figure}

\subsubsection{Zbiór xy-004.csv}
Analiza czwartego zbioru danych:
\begin{itemize}
    \item MSE: 2989424.6945
    \item RMSE: 1728.9953
    \item MAE: 1460.5613
    \item R²: 0.0015
\end{itemize}

\begin{figure}[H]
    \centering
    \begin{minipage}{0.45\textwidth}
        \includegraphics[width=\linewidth]{Screenshot 2025-03-24 at 22.15.21.png}
        \caption{Historia funkcji straty dla xy-004}
    \end{minipage}
    \hfill
    \begin{minipage}{0.45\textwidth}
        \includegraphics[width=\linewidth]{Screenshot 2025-03-24 at 22.15.23.png}
        \caption{Regresja i punkty dla xy-004}
    \end{minipage}
\end{figure}

\subsubsection{Zbiór xy-005.csv}
Wyniki dla piątego zbioru danych:
\begin{itemize}
    \item MSE: 1314.8936
    \item RMSE: 36.2615
    \item MAE: 27.3464
    \item R²: 0.2764
\end{itemize}

\begin{figure}[H]
    \centering
    \begin{minipage}{0.45\textwidth}
        \includegraphics[width=\linewidth]{Screenshot 2025-03-24 at 22.15.43.png}
        \caption{Historia funkcji straty dla xy-005}
    \end{minipage}
    \hfill
    \begin{minipage}{0.45\textwidth}
        \includegraphics[width=\linewidth]{Screenshot 2025-03-24 at 22.15.44.png}
        \caption{Regresja i punkty dla xy-005}
    \end{minipage}
\end{figure}

\subsubsection{Zbiór xy-006.csv}
Analiza szóstego zbioru danych:
\begin{itemize}
    \item MSE: 9.0792
    \item RMSE: 3.0132
    \item MAE: 2.6499
    \item R²: ≈ 0
\end{itemize}

\begin{figure}[H]
    \centering
    \begin{minipage}{0.45\textwidth}
        \includegraphics[width=\linewidth]{Screenshot 2025-03-24 at 22.16.01.png}
        \caption{Historia funkcji straty dla xy-006}
    \end{minipage}
    \hfill
    \begin{minipage}{0.45\textwidth}
        \includegraphics[width=\linewidth]{Screenshot 2025-03-24 at 22.16.12.png}
        \caption{Regresja i punkty dla xy-006}
    \end{minipage}
\end{figure}

\subsubsection{Zbiór xy-007.csv}
Wyniki dla siódmego zbioru danych:
\begin{itemize}
    \item MSE: 6.0685
    \item RMSE: 2.4634
    \item MAE: 2.0917
    \item R²: ≈ 0
\end{itemize}

\begin{figure}[H]
    \centering
    \begin{minipage}{0.45\textwidth}
        \includegraphics[width=\linewidth]{Screenshot 2025-03-24 at 22.16.36.png}
        \caption{Historia funkcji straty dla xy-007}
    \end{minipage}
    \hfill
    \begin{minipage}{0.45\textwidth}
        \includegraphics[width=\linewidth]{Screenshot 2025-03-24 at 22.16.38.png}
        \caption{Regresja i punkty dla xy-007}
    \end{minipage}
\end{figure}

\subsubsection{Zbiór xy-008.csv}
Analiza ósmego zbioru danych:
\begin{itemize}
    \item MSE: 32.7178
    \item RMSE: 5.7200
    \item MAE: 4.8248
    \item R²: ≈ 0
\end{itemize}

\begin{figure}[H]
    \centering
    \begin{minipage}{0.45\textwidth}
        \includegraphics[width=\linewidth]{Screenshot 2025-03-24 at 22.16.53.png}
        \caption{Historia funkcji straty dla xy-008}
    \end{minipage}
    \hfill
    \begin{minipage}{0.45\textwidth}
        \includegraphics[width=\linewidth]{Screenshot 2025-03-24 at 22.16.55.png}
        \caption{Regresja i punkty dla xy-008}
    \end{minipage}
\end{figure}

\subsubsection{Zbiór xy-009.csv}
Wyniki dla dziewiątego zbioru danych:
\begin{itemize}
    \item MSE: 19.0951
    \item RMSE: 4.3698
    \item MAE: 3.7244
    \item R²: ≈ 0
\end{itemize}

\begin{figure}[H]
    \centering
    \begin{minipage}{0.45\textwidth}
        \includegraphics[width=\linewidth]{Screenshot 2025-03-24 at 22.17.09.png}
        \caption{Historia funkcji straty dla xy-009}
    \end{minipage}
    \hfill
    \begin{minipage}{0.45\textwidth}
        \includegraphics[width=\linewidth]{Screenshot 2025-03-24 at 22.17.11.png}
        \caption{Regresja i punkty dla xy-009}
    \end{minipage}
\end{figure}

\subsubsection{Zbiór xy-010.csv}
Analiza dziesiątego zbioru danych:
\begin{itemize}
    \item MSE: 25.4716
    \item RMSE: 5.0469
    \item MAE: 4.2663
    \item R²: ≈ 0
\end{itemize}

\begin{figure}[H]
    \centering
    \begin{minipage}{0.45\textwidth}
        \includegraphics[width=\linewidth]{Screenshot 2025-03-24 at 22.17.51.png}
        \caption{Historia funkcji straty dla xy-010}
    \end{minipage}
    \hfill
    \begin{minipage}{0.45\textwidth}
        \includegraphics[width=\linewidth]{Screenshot 2025-03-24 at 22.17.54.png}
        \caption{Regresja i punkty dla xy-010}
    \end{minipage}
\end{figure}

\section{Porównanie xy-002 i xy-004}

\subsection{xy-002}
\begin{verbatim}
                            OLS Regression Results                            
==============================================================================
Dep. Variable:                      y   R-squared:                       0.933
Model:                            OLS   Adj. R-squared:                  0.933
Method:                 Least Squares   F-statistic:                     1370.
Date:                Mon, 24 Mar 2025   Prob (F-statistic):           2.10e-59
Time:                        23:08:26   Log-Likelihood:                -711.10
No. Observations:                 100   AIC:                             1426.
Df Residuals:                      98   BIC:                             1431.
Df Model:                           1                                         
Covariance Type:            nonrobust                                         
==============================================================================
                 coef    std err          t      P>|t|      [0.025      0.975]
------------------------------------------------------------------------------
const        652.8037     60.627     10.768      0.000     532.491     773.116
x1           -74.3271      2.008    -37.012      0.000     -78.312     -70.342
==============================================================================
Omnibus:                       11.135   Durbin-Watson:                   0.062
Prob(Omnibus):                  0.004   Jarque-Bera (JB):                7.706
Skew:                          -0.547   Prob(JB):                       0.0212
Kurtosis:                       2.191   Cond. No.                         61.2
==============================================================================    
\end{verbatim}

\begin{itemize}
    \item  Jaką wartość ma współczynnik determinacji?
    
Współczynnik determinacji (R-squared) wynosi 0.933, czyli 93.3\% zmienności zmiennej zależnej jest wyjaśnione przez model.

    \item Jaki jest błąd standardowy wyznaczonych współczynników?
    
Błędy standardowe (std err) dla współczynników wynoszą:
    Dla wyrazu wolnego: 60.627
    Dla zmiennej x1: 2.008

    \item W jakim zakresie mieszczą się z 95\% wiarygodnością?
    
95\% przedziały ufności (widoczne w kolumnach [0.025] i [0.975]):
    Dla wyrazu wolnego: od 532.491 do 773.116
    Dla zmiennej x1: od -78.312 do -70.342

\end{itemize}


\subsection{xy-004}
\begin{verbatim}
                            OLS Regression Results                            
==============================================================================
Dep. Variable:                      y   R-squared:                       0.002
Model:                            OLS   Adj. R-squared:                 -0.009
Method:                 Least Squares   F-statistic:                    0.1537
Date:                Mon, 24 Mar 2025   Prob (F-statistic):              0.696
Time:                        23:23:17   Log-Likelihood:                -887.42
No. Observations:                 100   AIC:                             1779.
Df Residuals:                      98   BIC:                             1784.
Df Model:                           1                                         
Covariance Type:            nonrobust                                         
==============================================================================
                 coef    std err          t      P>|t|      [0.025      0.975]
------------------------------------------------------------------------------
const       4124.0872    352.826     11.689      0.000    3423.916    4824.258
x1             4.8670     12.415      0.392      0.696     -19.771      29.505
==============================================================================
Omnibus:                        9.293   Durbin-Watson:                   0.013
Prob(Omnibus):                  0.010   Jarque-Bera (JB):                8.946
Skew:                          -0.671   Prob(JB):                       0.0114
Kurtosis:                       2.414   Cond. No.                         57.5
==============================================================================
\end{verbatim}

\begin{itemize}
    \item  Jaką wartość ma współczynnik determinacji?
    
Współczynnik determinacji (R-squared) jest bardzo niski i wynosi 0.002, co oznacza, że tylko 0.2\% zmienności zmiennej zależnej jest wyjaśnione przez model. To wskazuje na bardzo słabe dopasowanie modelu.

    \item Jaki jest błąd standardowy wyznaczonych współczynników?
    
Błędy standardowe (std err) dla współczynników wynoszą:
    Dla wyrazu wolnego (stałej): 352.826
    Dla zmiennej x1: 12.415

    \item W jakim zakresie mieszczą się z 95\% wiarygodnością?
    
95\% przedziały ufności (widoczne w kolumnach [0.025] i [0.975]):
    Dla wyrazu wolnego: od 3423.916 do 4824.258
    Dla x1: od -19.771 do 29.505

    \item Jak zinterpretujesz fakt, że dolna granica przedziału ufności dla współczynnika x1 to liczba ujemna, a górna to dodatnia?

Przedział ufności dla współczynnika x1 zawierający zarówno wartości ujemne jak i dodatnie oznacza, że nie możemy stwierdzić, że zmienna x1 ma jakikolwiek istotny wpływ na zmienną zależną.

\end{itemize}

\begin{figure}[H]
    \centering
    \begin{minipage}{0.45\textwidth}
        \includegraphics[width=\linewidth]{reg.png}
        \caption{Skrajne przebiegi dla xy-002}
    \end{minipage}
    \hfill
    \begin{minipage}{0.45\textwidth}
        \includegraphics[width=\linewidth]{image.png}
        \caption{Skrajne przebiegi dla xy-004}
    \end{minipage}
\end{figure}

\end{document}