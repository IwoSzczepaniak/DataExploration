\documentclass{article}

% Language setting
% Replace `english' with e.g. `spanish' to change the document language
\usepackage[polish]{babel}

% Set page size and margins
% Replace `letterpaper' with `a4paper' for UK/EU standard size
\usepackage[a4paper,top=2cm,bottom=2cm,left=3cm,right=3cm,marginparwidth=1.75cm]{geometry}
\usepackage[T1]{fontenc}

% Useful packages
\usepackage{amsmath}
\usepackage{graphicx}
\usepackage[colorlinks=true, allcolors=blue]{hyperref}
\usepackage{float}

\title{Eksploracja Danych - zadanie 3}
\author{Iwo Szczepaniak}

\begin{document}
\maketitle

\section{Przetwarzanie zbioru xy-001}
\begin{verbatim}
Resource name: xy-002.csv
Data with polynomial features:
+--------+----------+------------------+
|       X|         Y|                X2|
+--------+----------+------------------+
|0.411194|-59.938274|    0.169080505636|
|0.549662|-72.006761|    0.302128314244|
|0.860093|-68.979336|    0.739759968649|
| 1.27504|  32.07157|1.6257270015999998|
|2.202931|-91.531894|    4.852904990761|
+--------+----------+------------------+
only showing top 5 rows

root
 |-- X: double (nullable = true)
 |-- Y: double (nullable = true)
 |-- X2: double (nullable = true)
\end{verbatim}

\subsection{Analiza wizualna}
\begin{figure}[H]
    \centering
    \begin{minipage}{0.45\textwidth}
        \includegraphics[width=\linewidth]{1.png}
        \caption{Wykres 1}
    \end{minipage}
    \hfill
    \begin{minipage}{0.45\textwidth}
        \includegraphics[width=\linewidth]{2.png}
        \caption{Wykres 2}
    \end{minipage}
\end{figure}

\subsection{Wszystkie pliki}
\subsubsection{Analiza pierwszego zestawu}
\begin{figure}[H]
    \centering
    \begin{minipage}{0.45\textwidth}
        \includegraphics[width=\linewidth]{Screenshot 2025-03-27 at 15.37.32.png}
        \caption{Wykres 3}
    \end{minipage}
    \hfill
    \begin{minipage}{0.45\textwidth}
        \includegraphics[width=\linewidth]{Screenshot 2025-03-27 at 15.37.43.png}
        \caption{Wykres 4}
    \end{minipage}
\end{figure}

\subsubsection{Analiza drugiego zestawu}
\begin{figure}[H]
    \centering
    \begin{minipage}{0.45\textwidth}
        \includegraphics[width=\linewidth]{3.png}
        \caption{Wykres 5}
    \end{minipage}
    \hfill
    \begin{minipage}{0.45\textwidth}
        \includegraphics[width=\linewidth]{Screenshot 2025-03-27 at 15.38.14.png}
        \caption{Wykres 6}
    \end{minipage}
\end{figure}

\subsubsection{Analiza trzeciego zestawu}
\begin{figure}[H]
    \centering
    \begin{minipage}{0.45\textwidth}
        \includegraphics[width=\linewidth]{Screenshot 2025-03-27 at 15.38.48.png}
        \caption{Wykres 7}
    \end{minipage}
    \hfill
    \begin{minipage}{0.45\textwidth}
        \includegraphics[width=\linewidth]{Screenshot 2025-03-27 at 15.38.57.png}
        \caption{Wykres 8}
    \end{minipage}
\end{figure}

\subsubsection{Analiza czwartego zestawu}
\begin{figure}[H]
    \centering
    \begin{minipage}{0.45\textwidth}
        \includegraphics[width=\linewidth]{Screenshot 2025-03-27 at 15.39.22.png}
        \caption{Wykres 9}
    \end{minipage}
    \hfill
    \begin{minipage}{0.45\textwidth}
        \includegraphics[width=\linewidth]{Screenshot 2025-03-27 at 15.39.30.png}
        \caption{Wykres 10}
    \end{minipage}
\end{figure}

\subsubsection{Analiza piątego zestawu}
\begin{figure}[H]
    \centering
    \begin{minipage}{0.45\textwidth}
        \includegraphics[width=\linewidth]{Screenshot 2025-03-27 at 15.40.16.png}
        \caption{Wykres 11}
    \end{minipage}
    \hfill
    \begin{minipage}{0.45\textwidth}
        \includegraphics[width=\linewidth]{Screenshot 2025-03-27 at 15.40.48.png}
        \caption{Wykres 12}
    \end{minipage}
\end{figure}

\subsubsection{Analiza szóstego zestawu}
\begin{figure}[H]
    \centering
    \begin{minipage}{0.45\textwidth}
        \includegraphics[width=\linewidth]{Screenshot 2025-03-27 at 15.41.01.png}
        \caption{Wykres 13}
    \end{minipage}
    \hfill
    \begin{minipage}{0.45\textwidth}
        \includegraphics[width=\linewidth]{Screenshot 2025-03-27 at 15.41.48.png}
        \caption{Wykres 14}
    \end{minipage}
\end{figure}

\subsubsection{Analiza siódmego zestawu}
\begin{figure}[H]
    \centering
    \begin{minipage}{0.45\textwidth}
        \includegraphics[width=\linewidth]{Screenshot 2025-03-27 at 15.41.24.png}
        \caption{Wykres 15}
    \end{minipage}
    \hfill
    \begin{minipage}{0.45\textwidth}
        \includegraphics[width=\linewidth]{Screenshot 2025-03-27 at 15.42.01.png}
        \caption{Wykres 16}
    \end{minipage}
\end{figure}

\subsubsection{Analiza ósmego zestawu}
\begin{figure}[H]
    \centering
    \begin{minipage}{0.45\textwidth}
        \includegraphics[width=\linewidth]{4e.png}
        \caption{Wykres 17}
    \end{minipage}
    \hfill
    \begin{minipage}{0.45\textwidth}
        \includegraphics[width=\linewidth]{5.png}
        \caption{Wykres 18}
    \end{minipage}
\end{figure}

\subsubsection{Analiza dziewiątego zestawu}
\begin{figure}[H]
    \centering
    \begin{minipage}{0.45\textwidth}
        \includegraphics[width=\linewidth]{1age.png}
        \caption{Wykres 19}
    \end{minipage}
    \hfill
    \begin{minipage}{0.45\textwidth}
        \includegraphics[width=\linewidth]{12e.png}
        \caption{Wykres 20}
    \end{minipage}
\end{figure}

\subsubsection{Analiza dziesiątego zestawu}
\begin{figure}[H]
    \centering
    \begin{minipage}{0.45\textwidth}
        \includegraphics[width=\linewidth]{Screenshot 2025-03-27 at 15.43.09.png}
        \caption{Wykres 21}
    \end{minipage}
    \hfill
    \begin{minipage}{0.45\textwidth}
        \includegraphics[width=\linewidth]{Screenshot 2025-03-27 at 15.43.23.png}
        \caption{Wykres 22}
    \end{minipage}
\end{figure}

\subsection{Poprawienie dopasowania}
Ponieważ w poprzednim podpunkcie ustawiłem liczbę iteracji na 100, to wyniki dla zbioru 4 były bardzo dobre, jednak usunięcie regularyzacji umożliwia dopasowanie idealne - wtedy dochodzi do przeuczenia.

\begin{figure}[H]
    \centering
    \begin{minipage}{0.45\textwidth}
        \includegraphics[width=\linewidth]{asdage.png}
        \caption{Wykres funkcji straty}
    \end{minipage}
    \hfill
    \begin{minipage}{0.45\textwidth}
        \includegraphics[width=\linewidth]{1231age.png}
        \caption{Dopasowanie modelu}
    \end{minipage}
\end{figure}

\begin{figure}[H]
    \centering
    \begin{minipage}{0.45\textwidth}
        \includegraphics[width=\linewidth]{imagasde.png}
        \caption{Wykres funkcji straty po zwiększeniu iteracji}
    \end{minipage}
    \hfill
    \begin{minipage}{0.45\textwidth}
        \includegraphics[width=\linewidth]{image.png}
        \caption{Dopasowanie modelu po zwiększeniu iteracji}
    \end{minipage}
\end{figure}

\end{document}